% Options for packages loaded elsewhere
\PassOptionsToPackage{unicode}{hyperref}
\PassOptionsToPackage{hyphens}{url}
\PassOptionsToPackage{dvipsnames,svgnames,x11names}{xcolor}
%
\documentclass[
  letterpaper,
  DIV=11,
  numbers=noendperiod]{scrartcl}

\usepackage{amsmath,amssymb}
\usepackage{iftex}
\ifPDFTeX
  \usepackage[T1]{fontenc}
  \usepackage[utf8]{inputenc}
  \usepackage{textcomp} % provide euro and other symbols
\else % if luatex or xetex
  \usepackage{unicode-math}
  \defaultfontfeatures{Scale=MatchLowercase}
  \defaultfontfeatures[\rmfamily]{Ligatures=TeX,Scale=1}
\fi
\usepackage{lmodern}
\ifPDFTeX\else  
    % xetex/luatex font selection
\fi
% Use upquote if available, for straight quotes in verbatim environments
\IfFileExists{upquote.sty}{\usepackage{upquote}}{}
\IfFileExists{microtype.sty}{% use microtype if available
  \usepackage[]{microtype}
  \UseMicrotypeSet[protrusion]{basicmath} % disable protrusion for tt fonts
}{}
\makeatletter
\@ifundefined{KOMAClassName}{% if non-KOMA class
  \IfFileExists{parskip.sty}{%
    \usepackage{parskip}
  }{% else
    \setlength{\parindent}{0pt}
    \setlength{\parskip}{6pt plus 2pt minus 1pt}}
}{% if KOMA class
  \KOMAoptions{parskip=half}}
\makeatother
\usepackage{xcolor}
\setlength{\emergencystretch}{3em} % prevent overfull lines
\setcounter{secnumdepth}{-\maxdimen} % remove section numbering
% Make \paragraph and \subparagraph free-standing
\ifx\paragraph\undefined\else
  \let\oldparagraph\paragraph
  \renewcommand{\paragraph}[1]{\oldparagraph{#1}\mbox{}}
\fi
\ifx\subparagraph\undefined\else
  \let\oldsubparagraph\subparagraph
  \renewcommand{\subparagraph}[1]{\oldsubparagraph{#1}\mbox{}}
\fi


\providecommand{\tightlist}{%
  \setlength{\itemsep}{0pt}\setlength{\parskip}{0pt}}\usepackage{longtable,booktabs,array}
\usepackage{calc} % for calculating minipage widths
% Correct order of tables after \paragraph or \subparagraph
\usepackage{etoolbox}
\makeatletter
\patchcmd\longtable{\par}{\if@noskipsec\mbox{}\fi\par}{}{}
\makeatother
% Allow footnotes in longtable head/foot
\IfFileExists{footnotehyper.sty}{\usepackage{footnotehyper}}{\usepackage{footnote}}
\makesavenoteenv{longtable}
\usepackage{graphicx}
\makeatletter
\def\maxwidth{\ifdim\Gin@nat@width>\linewidth\linewidth\else\Gin@nat@width\fi}
\def\maxheight{\ifdim\Gin@nat@height>\textheight\textheight\else\Gin@nat@height\fi}
\makeatother
% Scale images if necessary, so that they will not overflow the page
% margins by default, and it is still possible to overwrite the defaults
% using explicit options in \includegraphics[width, height, ...]{}
\setkeys{Gin}{width=\maxwidth,height=\maxheight,keepaspectratio}
% Set default figure placement to htbp
\makeatletter
\def\fps@figure{htbp}
\makeatother

\KOMAoption{captions}{tableheading}
\makeatletter
\@ifpackageloaded{caption}{}{\usepackage{caption}}
\AtBeginDocument{%
\ifdefined\contentsname
  \renewcommand*\contentsname{Table of contents}
\else
  \newcommand\contentsname{Table of contents}
\fi
\ifdefined\listfigurename
  \renewcommand*\listfigurename{List of Figures}
\else
  \newcommand\listfigurename{List of Figures}
\fi
\ifdefined\listtablename
  \renewcommand*\listtablename{List of Tables}
\else
  \newcommand\listtablename{List of Tables}
\fi
\ifdefined\figurename
  \renewcommand*\figurename{Figure}
\else
  \newcommand\figurename{Figure}
\fi
\ifdefined\tablename
  \renewcommand*\tablename{Table}
\else
  \newcommand\tablename{Table}
\fi
}
\@ifpackageloaded{float}{}{\usepackage{float}}
\floatstyle{ruled}
\@ifundefined{c@chapter}{\newfloat{codelisting}{h}{lop}}{\newfloat{codelisting}{h}{lop}[chapter]}
\floatname{codelisting}{Listing}
\newcommand*\listoflistings{\listof{codelisting}{List of Listings}}
\makeatother
\makeatletter
\makeatother
\makeatletter
\@ifpackageloaded{caption}{}{\usepackage{caption}}
\@ifpackageloaded{subcaption}{}{\usepackage{subcaption}}
\makeatother
\ifLuaTeX
  \usepackage{selnolig}  % disable illegal ligatures
\fi
\usepackage{bookmark}

\IfFileExists{xurl.sty}{\usepackage{xurl}}{} % add URL line breaks if available
\urlstyle{same} % disable monospaced font for URLs
\hypersetup{
  pdftitle={TCA-3 Survey},
  pdfauthor={Ben McQuaile, Dylan Dockery, Luke Cassidy, Emily McCormack},
  colorlinks=true,
  linkcolor={blue},
  filecolor={Maroon},
  citecolor={Blue},
  urlcolor={Blue},
  pdfcreator={LaTeX via pandoc}}

\title{TCA-3 Survey}
\author{Ben McQuaile, Dylan Dockery, Luke Cassidy, Emily McCormack}
\date{}

\begin{document}
\maketitle

\subsection{Overview}\label{overview}

This is our document detailing the work conducted by our team,
VisionWeavers in surveying participants who agreed to take part in
testing our game Shades of Solitude.

\subsection{Survey Construction}\label{survey-construction}

\subsubsection{Categories of Questions}\label{categories-of-questions}

\paragraph{Demographics}\label{demographics}

These survey questions are designed to help understand and segment our
audience by characteristics such as age, gender and how much they play
games. This helps to inform us make better decisions when designing the
game. For example, more experienced gamers might find it easier.

\begin{itemize}
\item
  \textbf{What Gender do you identify as?}

  \begin{itemize}
  \tightlist
  \item
    Male
  \item
    Female
  \item
    Other
  \end{itemize}
\item
  \textbf{What is your Age Group?}

  \begin{itemize}
  \tightlist
  \item
    18-24
  \item
    25-34
  \item
    35-44
  \item
    45-54
  \item
    55-64
  \item
    65+
  \end{itemize}
\item
  \textbf{How many hours would you play in games in a given week?}

  \begin{itemize}
  \tightlist
  \item
    1-5 Hours
  \item
    6-15 Hours
  \item
    16-30 Hours
  \item
    31-50 Hours
  \item
    51+ Hours
  \end{itemize}
\end{itemize}

\paragraph{Mechanics}\label{mechanics}

These questions are designed to help us improve the playability and
overall enjoyment of our game.

\begin{itemize}
\tightlist
\item
  \textbf{How intuitive did you find the player controls?}

  \begin{itemize}
  \tightlist
  \item
    Very intuitive
  \item
    Somewhat intuitive
  \item
    Neutral
  \item
    Somewhat unintuitive
  \item
    Very unintuitive
  \end{itemize}
\item
  \textbf{Did you experience any difficulties with the player's
  controls?}

  \begin{itemize}
  \tightlist
  \item
    Yes (Please Specify)
  \item
    No
  \end{itemize}
\item
  \textbf{How challenging did you find the puzzles in the game?}

  \begin{itemize}
  \tightlist
  \item
    Too Easy
  \item
    Just Right
  \item
    Neutral
  \item
    Very Difficult
  \item
    Frustratingly Difficult
  \end{itemize}
\item
  \textbf{Was there a particular puzzle that you felt was too difficult
  or too easy?}

  \begin{itemize}
  \tightlist
  \item
    Too Easy (Please Describe)
  \item
    Too Difficult (Please Describe)
  \item
    No
  \end{itemize}
\end{itemize}

\paragraph{Narrative}\label{narrative}

These questions are designed to help us understand that the themes of
our game come across to the player and which elements get these across
best. It also helps identify areas that are weak in portraying the theme
and need to be improved.

\begin{itemize}
\item
  \textbf{How well do you think the game communicated its theme?}

  \begin{itemize}
  \tightlist
  \item
    Very clearly
  \item
    Somewhat clearly
  \item
    Neutral
  \item
    Somewhat unclearly
  \item
    Very unclear
  \end{itemize}
\item
  \textbf{Which elements of the game helped you understand the theme?
  (Select all that apply)}

  \begin{itemize}
  \tightlist
  \item
    Visuals
  \item
    Puzzles
  \item
    Audio
  \item
    Text/dialogue
  \item
    Other
  \end{itemize}
\end{itemize}

\paragraph{Audio}\label{audio}

These questions help determine the contribution of audio to the game's
atmosphere,identifying effective and ineffective elements. Being able to
distinguish on effective and ineffective elements help us shape the
soundscape of the game to better suit the atmosphere. We can gather
insights on how much work is needed on the audio for future releases.

\begin{itemize}
\item
  \textbf{How did you feel about the audio quality and its contribution
  to the game atmosphere?}

  \begin{itemize}
  \tightlist
  \item
    Enhanced the experience greatly
  \item
    Somewhat enhanced the experience
  \item
    Did not effect experience
  \item
    Somewhat detracted from the experience
  \item
    Greatly detracted from the experience
  \end{itemize}
\item
  \textbf{Were there any audio elements you found particularly
  effective?}

  \begin{itemize}
  \tightlist
  \item
    Yes (Please specify below)
  \item
    No
  \end{itemize}
\item
  \textbf{Were there any audio elements you found particularly
  ineffective?}

  \begin{itemize}
  \tightlist
  \item
    Yes (Please specify below)
  \item
    No
  \end{itemize}
\end{itemize}

\paragraph{General}\label{general}

\begin{itemize}
\item
  \textbf{On a scale of 1-10, how likely are you to recommend ``Shades
  of Solitude'' to others}

  \begin{itemize}
  \tightlist
  \item
    Range 1-10 (Not likely at all to Extremely likely)
  \end{itemize}
\end{itemize}

\subsubsection{Survey Analysis}\label{survey-analysis}

The data (N=\texttt{r\ nrow(survey\_data)}) was collected from a
convenience sample of participants who were asked to complete the survey
online. The purpose of the survey was to gather information on a number
of categories relating to game mechanics, visuals, audio and narrative.



\end{document}
