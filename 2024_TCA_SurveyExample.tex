% Options for packages loaded elsewhere
\PassOptionsToPackage{unicode}{hyperref}
\PassOptionsToPackage{hyphens}{url}
\PassOptionsToPackage{dvipsnames,svgnames,x11names}{xcolor}
%
\documentclass[
  letterpaper,
  DIV=11,
  numbers=noendperiod]{scrartcl}

\usepackage{amsmath,amssymb}
\usepackage{iftex}
\ifPDFTeX
  \usepackage[T1]{fontenc}
  \usepackage[utf8]{inputenc}
  \usepackage{textcomp} % provide euro and other symbols
\else % if luatex or xetex
  \usepackage{unicode-math}
  \defaultfontfeatures{Scale=MatchLowercase}
  \defaultfontfeatures[\rmfamily]{Ligatures=TeX,Scale=1}
\fi
\usepackage{lmodern}
\ifPDFTeX\else  
    % xetex/luatex font selection
\fi
% Use upquote if available, for straight quotes in verbatim environments
\IfFileExists{upquote.sty}{\usepackage{upquote}}{}
\IfFileExists{microtype.sty}{% use microtype if available
  \usepackage[]{microtype}
  \UseMicrotypeSet[protrusion]{basicmath} % disable protrusion for tt fonts
}{}
\makeatletter
\@ifundefined{KOMAClassName}{% if non-KOMA class
  \IfFileExists{parskip.sty}{%
    \usepackage{parskip}
  }{% else
    \setlength{\parindent}{0pt}
    \setlength{\parskip}{6pt plus 2pt minus 1pt}}
}{% if KOMA class
  \KOMAoptions{parskip=half}}
\makeatother
\usepackage{xcolor}
\setlength{\emergencystretch}{3em} % prevent overfull lines
\setcounter{secnumdepth}{-\maxdimen} % remove section numbering
% Make \paragraph and \subparagraph free-standing
\ifx\paragraph\undefined\else
  \let\oldparagraph\paragraph
  \renewcommand{\paragraph}[1]{\oldparagraph{#1}\mbox{}}
\fi
\ifx\subparagraph\undefined\else
  \let\oldsubparagraph\subparagraph
  \renewcommand{\subparagraph}[1]{\oldsubparagraph{#1}\mbox{}}
\fi


\providecommand{\tightlist}{%
  \setlength{\itemsep}{0pt}\setlength{\parskip}{0pt}}\usepackage{longtable,booktabs,array}
\usepackage{calc} % for calculating minipage widths
% Correct order of tables after \paragraph or \subparagraph
\usepackage{etoolbox}
\makeatletter
\patchcmd\longtable{\par}{\if@noskipsec\mbox{}\fi\par}{}{}
\makeatother
% Allow footnotes in longtable head/foot
\IfFileExists{footnotehyper.sty}{\usepackage{footnotehyper}}{\usepackage{footnote}}
\makesavenoteenv{longtable}
\usepackage{graphicx}
\makeatletter
\def\maxwidth{\ifdim\Gin@nat@width>\linewidth\linewidth\else\Gin@nat@width\fi}
\def\maxheight{\ifdim\Gin@nat@height>\textheight\textheight\else\Gin@nat@height\fi}
\makeatother
% Scale images if necessary, so that they will not overflow the page
% margins by default, and it is still possible to overwrite the defaults
% using explicit options in \includegraphics[width, height, ...]{}
\setkeys{Gin}{width=\maxwidth,height=\maxheight,keepaspectratio}
% Set default figure placement to htbp
\makeatletter
\def\fps@figure{htbp}
\makeatother

\usepackage{booktabs}
\usepackage{longtable}
\usepackage{array}
\usepackage{multirow}
\usepackage{wrapfig}
\usepackage{float}
\usepackage{colortbl}
\usepackage{pdflscape}
\usepackage{tabu}
\usepackage{threeparttable}
\usepackage{threeparttablex}
\usepackage[normalem]{ulem}
\usepackage{makecell}
\usepackage{xcolor}
\KOMAoption{captions}{tableheading}
\makeatletter
\@ifpackageloaded{caption}{}{\usepackage{caption}}
\AtBeginDocument{%
\ifdefined\contentsname
  \renewcommand*\contentsname{Table of contents}
\else
  \newcommand\contentsname{Table of contents}
\fi
\ifdefined\listfigurename
  \renewcommand*\listfigurename{List of Figures}
\else
  \newcommand\listfigurename{List of Figures}
\fi
\ifdefined\listtablename
  \renewcommand*\listtablename{List of Tables}
\else
  \newcommand\listtablename{List of Tables}
\fi
\ifdefined\figurename
  \renewcommand*\figurename{Figure}
\else
  \newcommand\figurename{Figure}
\fi
\ifdefined\tablename
  \renewcommand*\tablename{Table}
\else
  \newcommand\tablename{Table}
\fi
}
\@ifpackageloaded{float}{}{\usepackage{float}}
\floatstyle{ruled}
\@ifundefined{c@chapter}{\newfloat{codelisting}{h}{lop}}{\newfloat{codelisting}{h}{lop}[chapter]}
\floatname{codelisting}{Listing}
\newcommand*\listoflistings{\listof{codelisting}{List of Listings}}
\makeatother
\makeatletter
\makeatother
\makeatletter
\@ifpackageloaded{caption}{}{\usepackage{caption}}
\@ifpackageloaded{subcaption}{}{\usepackage{subcaption}}
\makeatother
\ifLuaTeX
  \usepackage{selnolig}  % disable illegal ligatures
\fi
\usepackage{bookmark}

\IfFileExists{xurl.sty}{\usepackage{xurl}}{} % add URL line breaks if available
\urlstyle{same} % disable monospaced font for URLs
\hypersetup{
  pdftitle={2024 - TCA-3 - Survey},
  pdfauthor={Dylan Dockery, Ben McQuaile, Luke Cassidy, Emily McCormack},
  colorlinks=true,
  linkcolor={blue},
  filecolor={Maroon},
  citecolor={Blue},
  urlcolor={Blue},
  pdfcreator={LaTeX via pandoc}}

\title{2024 - TCA-3 - Survey}
\author{Dylan Dockery, Ben McQuaile, Luke Cassidy, Emily McCormack}
\date{2024-05-06}

\begin{document}
\maketitle

\renewcommand*\contentsname{Table of contents}
{
\hypersetup{linkcolor=}
\setcounter{tocdepth}{3}
\tableofcontents
}
\subsection{Overview}\label{overview}

This is our document detailing the work conducted by our team,
VisionWeavers in surveying participants who agreed to take part in
testing our game Shades of Solitude.

\subsection{Survey Construction}\label{survey-construction}

\subsubsection{Categories of Questions}\label{categories-of-questions}

\paragraph{Demographics}\label{demographics}

These survey questions are designed to help understand and segment our
audience by characteristics such as age, gender and how much they play
games. This helps to inform us make better decisions when designing the
game. For example, more experienced gamers might find it easier.

\begin{itemize}
\item
  \textbf{What Gender do you identify as?}

  \begin{itemize}
  \tightlist
  \item
    Male
  \item
    Female
  \item
    Other
  \end{itemize}
\item
  \textbf{What is your Age Group?}

  \begin{itemize}
  \tightlist
  \item
    18-24
  \item
    25-34
  \item
    35-44
  \item
    45-54
  \item
    55-64
  \item
    65+
  \end{itemize}
\item
  \textbf{How many hours would you play in games in a given week?}

  \begin{itemize}
  \tightlist
  \item
    1-5 Hours
  \item
    6-15 Hours
  \item
    16-30 Hours
  \item
    31-50 Hours
  \item
    51+ Hours
  \end{itemize}
\end{itemize}

\paragraph{Mechanics}\label{mechanics}

These questions are designed to help us improve the playability and
overall enjoyment of our game.

\begin{itemize}
\tightlist
\item
  \textbf{How intuitive did you find the player controls?}

  \begin{itemize}
  \tightlist
  \item
    Very intuitive
  \item
    Somewhat intuitive
  \item
    Neutral
  \item
    Somewhat unintuitive
  \item
    Very unintuitive
  \end{itemize}
\item
  \textbf{Did you experience any difficulties with the player's
  controls?}

  \begin{itemize}
  \tightlist
  \item
    Yes (Please Specify)
  \item
    No
  \end{itemize}
\item
  \textbf{How challenging did you find the puzzles in the game?}

  \begin{itemize}
  \tightlist
  \item
    Too Easy
  \item
    Just Right
  \item
    Neutral
  \item
    Very Difficult
  \item
    Frustratingly Difficult
  \end{itemize}
\item
  \textbf{Was there a particular puzzle that you felt was too difficult
  or too easy?}

  \begin{itemize}
  \tightlist
  \item
    Too Easy (Please Describe)
  \item
    Too Difficult (Please Describe)
  \item
    No
  \end{itemize}
\end{itemize}

\paragraph{Narrative}\label{narrative}

These questions are designed to help us understand that the themes of
our game come across to the player and which elements get these across
best. It also helps identify areas that are weak in portraying the theme
and need to be improved.

\begin{itemize}
\item
  \textbf{How well do you think the game communicated its theme?}

  \begin{itemize}
  \tightlist
  \item
    Very clearly
  \item
    Somewhat clearly
  \item
    Neutral
  \item
    Somewhat unclearly
  \item
    Very unclear
  \end{itemize}
\item
  \textbf{Which elements of the game helped you understand the theme?
  (Select all that apply)}

  \begin{itemize}
  \tightlist
  \item
    Visuals
  \item
    Puzzles
  \item
    Audio
  \item
    Text/dialogue
  \item
    Other
  \end{itemize}
\end{itemize}

\paragraph{Audio}\label{audio}

These questions help determine the contribution of audio to the game's
atmosphere,identifying effective and ineffective elements. Being able to
distinguish on effective and ineffective elements help us shape the
soundscape of the game to better suit the atmosphere. We can gather
insights on how much work is needed on the audio for future releases.

\begin{itemize}
\item
  \textbf{How did you feel about the audio quality and its contribution
  to the game atmosphere?}

  \begin{itemize}
  \tightlist
  \item
    Enhanced the experience greatly
  \item
    Somewhat enhanced the experience
  \item
    Did not effect experience
  \item
    Somewhat detracted from the experience
  \item
    Greatly detracted from the experience
  \end{itemize}
\item
  \textbf{Were there any audio elements you found particularly
  effective?}

  \begin{itemize}
  \tightlist
  \item
    Yes (Please specify below)
  \item
    No
  \end{itemize}
\item
  \textbf{Were there any audio elements you found particularly
  ineffective?}

  \begin{itemize}
  \tightlist
  \item
    Yes (Please specify below)
  \item
    No
  \end{itemize}
\end{itemize}

\paragraph{General}\label{general}

\begin{itemize}
\item
  \textbf{On a scale of 1-10, how likely are you to recommend ``Shades
  of Solitude'' to others}

  \begin{itemize}
  \tightlist
  \item
    Range 1-10 (Not likely at all to Extremely likely)
  \end{itemize}
\end{itemize}

\subsection{Survey Analysis}\label{survey-analysis}

The data (N=20) was collected from a convenience sample of participants
who were asked to complete the survey online. The purpose of the survey
was to gather information on a number of categories relating to game
mechanic, playabilty, and in-game content.

The results were gathered using Google forms and the data was stored in
a CSV file.

\subsection{Descriptive Statistics}\label{descriptive-statistics}

This section provides basic descriptive statistics for the collected
data. It includes measures of central tendency and variability.

\subsubsection{Recommendation
Likelihood}\label{recommendation-likelihood}

The Recommendation Likelihood data was collected on a scale of 0 (Not
Likely) to 10 (Very Likely). The data was analyzed to determine the
median and standard deviation.

The median Recommendation likelihood score was 7.5 with a standard
deviation of 1.73.

\includegraphics{2024_TCA_SurveyExample_files/figure-pdf/histogram likelihood-1.pdf}

The histogram in Fig 1.0 shows the distribution of recommendation
likelihood scores. The data appears to be left-skewed and unimodal.

\includegraphics{2024_TCA_SurveyExample_files/figure-pdf/boxplot likelihood-1.pdf}

The boxplot in Fig 2.0 shows the Recommendation Likelihood by
Atmosphere. This measures the likelihood of recommending the game based
on the atmosphere categories labeled as 1-5 respectively(1 - very
engaging, 2 - somewhat engaging)etc. A very engaging(1) atmosphere
displayed higher median scores between 8 and 9 with very little
variability. A somewhat engaging(2) atmosphere displayed more
variability with a lower median around 6 and included an outlier scoring
around 3. Somewhat disengaging(4) was represented by a single data point
which indicates limited data for this cater gory.

Outlier tells us that the atmosphere was interesting but it might not be
something they are massively interested in.

\subsection{Tests for
Normality(Recommendation)}\label{tests-for-normalityrecommendation}

The Recommendation Likelihood data was tested for normality using a QQ
plot and a Shapiro-Wilk test.

\includegraphics{2024_TCA_SurveyExample_files/figure-pdf/normality test 1-1.pdf}

The QQ Plot in Fig 3.0 shows that the data is normally distributed. To
confirm this, a Shapiro-Wilk test was conducted.

\begin{verbatim}

    Shapiro-Wilk normality test

data:  survey_data$Recommendation.Likelihood
W = 0.93994, p-value = 0.2392
\end{verbatim}

A Shapiro Wilk test was conducted on the Recommendation Likelihood data.
From the output obtained our Test Statistic(W) was = 0.94, this value
was close to 1 which indicates a fairly normal distribution with some
deviations. As a result we assume normality and \textbf{accept the null
hypothesis} as the p-value was 0.24 which was greater than the common
alpha level of 0.05.

\subsection{Tests for normality(Weekly
Playtime)}\label{tests-for-normalityweekly-playtime}

The Weekly playtime data was tested for normality using a QQ plot and a
Shapiro-Wilk test.

\includegraphics{2024_TCA_SurveyExample_files/figure-pdf/normality test 2-1.pdf}

The QQ Plot in Fig 3.1 shows that the data is not normally distributed.
To confirm this, a Shapiro-Wilk test was conducted.

\begin{verbatim}

    Shapiro-Wilk normality test

data:  hours_played
W = 0.85756, p-value = 0.007156
\end{verbatim}

A Shapiro Wilk test was conducted on the Weekly playtime data. From the
output obtained our Test Statistic(W) was = 0.86, this value is much
lower than 1 which indicates it does not follow normal distribution. As
a result we do not assume normality and leads us to \textbf{reject the
null hypothesis} as the p-value was 0.007156 which was less than the
common alpha level of 0.05.

\subsection{Tests for Normality(Intuitive
Controls)}\label{tests-for-normalityintuitive-controls}

The QQ Plot in Fig 3.2 shows that the data is normally distributed. To
confirm this, a Shapiro-Wilk test was conducted.

\includegraphics{2024_TCA_SurveyExample_files/figure-pdf/normailty test 3-1.pdf}

The QQ Plot in Fig 3.2 shows that the data is not normally distributed.
To confirm this, a Shapiro-Wilk test was conducted.

\begin{verbatim}

    Shapiro-Wilk normality test

data:  intuitive_controls
W = 0.71075, p-value = 5.287e-05
\end{verbatim}

A Shapiro Wilk test was conducted on the Weekly playtime data. From the
output obtained our Test Statistic(W) was = 0.71, this value is much
lower than 1 which indicates it does not follow normal distribution. As
a result we do not assume normality and leads us to \textbf{reject the
null hypothesis} as the p-value was 0.000005 which was less than the
common alpha level of 0.05.

\subsection{Correlation Analysis}\label{correlation-analysis}

\subsubsection{Hours per week vs Reccomendation Likelihood
(Numerical)}\label{hours-per-week-vs-reccomendation-likelihood-numerical}

This section investigates the relationships between two paired numerical
variables. Both numerical variables were examined for normality. While
Recommendation Likelihood was considered normal, Hours per week was not.
However we stilled use Pearson's correlation coefficient to assess the
strength of the relationship between these numerical variables.

\includegraphics{2024_TCA_SurveyExample_files/figure-pdf/correlation analysis-1.pdf}

The scatterplot above shows the relationship between the number of hours
played per week and the likelihood of recommending the game. The data
points are spread across the plot, indicating a potential relationship
between the two variables.

\begin{verbatim}

    Pearson's product-moment correlation

data:  hours_played and survey_data$Recommendation.Likelihood
t = 0.39335, df = 18, p-value = 0.6987
alternative hypothesis: true correlation is not equal to 0
95 percent confidence interval:
 -0.3651196  0.5138461
sample estimates:
       cor 
0.09231712 
\end{verbatim}

A correlation test between the number of hours played per week and the
likelihood of recommending the game was conducted. The results of the
correlation test indicate a non-significant relationship between the
number of hours played per week and the likelihood of purchasing in-game
content. The estimated correlation is 0.09.

Based on these results, there is no substantial evidence to conclude
that there is a meaningful or statistically significant correlation
between the number of hours played and the likelihood of recommending
the product.

\subsubsection{Hours Played vs Controls Intuitiveness
(Catergorical)}\label{hours-played-vs-controls-intuitiveness-catergorical}

In this section, we will examine the relationship between how intuitive
the controls were and amount of hours people play games in a week . We
will convert the categorical data to numerical data(done above) and then
calculate the correlation between the two variables. Both numerical
variables were examined for normality and are considered not normal.
However we still used Spearman's correlation coefficient to assess the
strength of the relationship between these categorical variables.

\begin{verbatim}
\end{verbatim}

\includegraphics{2024_TCA_SurveyExample_files/figure-pdf/scatterplot-1.pdf}

The scatterplot above shows the relationship between hours played and
the intuitiveness of the controls. The data points are spread across the
plot, indicating a potential relationship between the two variables.

\begin{verbatim}

    Spearman's rank correlation rho

data:  hours_played and intuitive_controls
S = 2038.2, p-value = 0.01565
alternative hypothesis: true rho is not equal to 0
sample estimates:
       rho 
-0.5324736 
\end{verbatim}

A correlation test between the hours played and the intuitiveness of the
controls to was conducted. The results of the correlation test indicate
a significant relationship between the hours played and the likelihood
of recommending the game to a friend. The estimated correlation is
-0.53.

The Spearman's rank correlation test shows a significant moderate
negative correlation between the hours played and the ratings of
intuitive controls. This could mean that players who spend more hours
playing rate the controls as more intuitive. This is due to the
labeling, for example( 1 - hours played = 1-5hrs and 1 - Intuitive
controls = Very intuitive)

\subsection{Appendix A - Tables}\label{appendix-a---tables}

\subsubsection{Categorical to Numerical}\label{categorical-to-numerical}

Table 1.0 below shows the results of converting the categorical data to
numerical data for the hours played and controls

\begin{longtable}[]{@{}rr@{}}
\caption{Table 1.0 - Categorical to Numerical Conversion}\tabularnewline
\toprule\noalign{}
Weekly Playtime & Intuitive Controls \\
\midrule\noalign{}
\endfirsthead
\toprule\noalign{}
Weekly Playtime & Intuitive Controls \\
\midrule\noalign{}
\endhead
\bottomrule\noalign{}
\endlastfoot
2 & 1 \\
2 & 1 \\
2 & 1 \\
2 & 1 \\
1 & 2 \\
1 & 1 \\
2 & 2 \\
1 & 3 \\
1 & 1 \\
3 & 1 \\
1 & 2 \\
3 & 1 \\
2 & 2 \\
1 & 2 \\
3 & 1 \\
3 & 1 \\
4 & 1 \\
4 & 1 \\
3 & 2 \\
1 & 3 \\
\end{longtable}

\subsection{Appendix B - Survey Data}\label{appendix-b---survey-data}

Table 2.0 below shows the raw survey data collected from the
participants.

\begin{longtable}[]{@{}
  >{\raggedright\arraybackslash}p{(\columnwidth - 24\tabcolsep) * \real{0.0230}}
  >{\raggedright\arraybackslash}p{(\columnwidth - 24\tabcolsep) * \real{0.0329}}
  >{\raggedright\arraybackslash}p{(\columnwidth - 24\tabcolsep) * \real{0.0526}}
  >{\raggedright\arraybackslash}p{(\columnwidth - 24\tabcolsep) * \real{0.0691}}
  >{\raggedright\arraybackslash}p{(\columnwidth - 24\tabcolsep) * \real{0.0954}}
  >{\raggedright\arraybackslash}p{(\columnwidth - 24\tabcolsep) * \real{0.0329}}
  >{\raggedright\arraybackslash}p{(\columnwidth - 24\tabcolsep) * \real{0.0625}}
  >{\raggedright\arraybackslash}p{(\columnwidth - 24\tabcolsep) * \real{0.0888}}
  >{\raggedright\arraybackslash}p{(\columnwidth - 24\tabcolsep) * \real{0.0789}}
  >{\raggedright\arraybackslash}p{(\columnwidth - 24\tabcolsep) * \real{0.0822}}
  >{\raggedright\arraybackslash}p{(\columnwidth - 24\tabcolsep) * \real{0.1678}}
  >{\raggedright\arraybackslash}p{(\columnwidth - 24\tabcolsep) * \real{0.1283}}
  >{\raggedleft\arraybackslash}p{(\columnwidth - 24\tabcolsep) * \real{0.0855}}@{}}
\caption{Raw Survey Data}\tabularnewline
\toprule\noalign{}
\begin{minipage}[b]{\linewidth}\raggedright
Gender
\end{minipage} & \begin{minipage}[b]{\linewidth}\raggedright
Age.Group
\end{minipage} & \begin{minipage}[b]{\linewidth}\raggedright
Weekly.playtime
\end{minipage} & \begin{minipage}[b]{\linewidth}\raggedright
Overall.Atmosphere
\end{minipage} & \begin{minipage}[b]{\linewidth}\raggedright
Favourite.Atmoshphere.Aspect
\end{minipage} & \begin{minipage}[b]{\linewidth}\raggedright
Nostalgic
\end{minipage} & \begin{minipage}[b]{\linewidth}\raggedright
Intuitive.Controls
\end{minipage} & \begin{minipage}[b]{\linewidth}\raggedright
Difficulties.with.Controls
\end{minipage} & \begin{minipage}[b]{\linewidth}\raggedright
Puzzle.Difficulty
\end{minipage} & \begin{minipage}[b]{\linewidth}\raggedright
Game.Theme.Communication
\end{minipage} & \begin{minipage}[b]{\linewidth}\raggedright
Theme.Providing.Elements
\end{minipage} & \begin{minipage}[b]{\linewidth}\raggedright
Audio.Contribution
\end{minipage} & \begin{minipage}[b]{\linewidth}\raggedleft
Recommendation.Likelihood
\end{minipage} \\
\midrule\noalign{}
\endfirsthead
\toprule\noalign{}
\begin{minipage}[b]{\linewidth}\raggedright
Gender
\end{minipage} & \begin{minipage}[b]{\linewidth}\raggedright
Age.Group
\end{minipage} & \begin{minipage}[b]{\linewidth}\raggedright
Weekly.playtime
\end{minipage} & \begin{minipage}[b]{\linewidth}\raggedright
Overall.Atmosphere
\end{minipage} & \begin{minipage}[b]{\linewidth}\raggedright
Favourite.Atmoshphere.Aspect
\end{minipage} & \begin{minipage}[b]{\linewidth}\raggedright
Nostalgic
\end{minipage} & \begin{minipage}[b]{\linewidth}\raggedright
Intuitive.Controls
\end{minipage} & \begin{minipage}[b]{\linewidth}\raggedright
Difficulties.with.Controls
\end{minipage} & \begin{minipage}[b]{\linewidth}\raggedright
Puzzle.Difficulty
\end{minipage} & \begin{minipage}[b]{\linewidth}\raggedright
Game.Theme.Communication
\end{minipage} & \begin{minipage}[b]{\linewidth}\raggedright
Theme.Providing.Elements
\end{minipage} & \begin{minipage}[b]{\linewidth}\raggedright
Audio.Contribution
\end{minipage} & \begin{minipage}[b]{\linewidth}\raggedleft
Recommendation.Likelihood
\end{minipage} \\
\midrule\noalign{}
\endhead
\bottomrule\noalign{}
\endlastfoot
Male & 18-24 & 6-15hrs & Very engaging & Audio Effects & Yes & Very
intuitive & No & Just Right & Somewhat clearly & Visuals & Enhanced the
experience greatly & 9 \\
Male & 18-24 & 6-15hrs & Somewhat engaging & Pacing of GamePlay & Yes &
Very intuitive & No & Just Right & Somewhat clearly & Puzzles & Enhanced
the experience greatly & 7 \\
Male & 18-24 & 6-15hrs & Very engaging & Audio Effects & Yes & Very
intuitive & No & Very difficult & Somewhat clearly & Text/dialogue &
Enhanced the experience greatly & 8 \\
Male & 25-34 & 6-15hrs & Very engaging & Visual Style & Yes & Very
intuitive & No & Just Right & Somewhat clearly & Text/dialogue &
Enhanced the experience greatly & 8 \\
Male & 35-44 & 1-5hrs & Somewhat engaging & Visual Style & Yes &
Somewhat intuitive & Yes(Please Specify) & Neutral & Somewhat clearly &
Text/dialogue & Enhanced the experience greatly & 7 \\
Male & 35-44 & 1-5hrs & Somewhat engaging & Visual Style & Yes & Very
intuitive & No & Too Easy & Somewhat clearly & Visuals & Enhanced the
experience greatly & 7 \\
Female & 18-24 & 6-15hrs & Somewhat engaging & Narrative Tone & Yes &
Somewhat intuitive & No & Just Right & Somewhat unclearly &
Visuals;Audio & Somewhat enhanced the experience & 6 \\
Female & 25-34 & 1-5hrs & Somewhat engaging & Narrative Tone & Yes &
Neutral & No & Neutral & Somewhat clearly & Visuals;Audio;Text/dialogue
& Enhanced the experience greatly & 7 \\
Male & 35-44 & 1-5hrs & Somewhat engaging & Visual Style & Yes & Very
intuitive & No & Too Easy & Very clearly & Visuals;Text/dialogue &
Somewhat enhanced the experience & 9 \\
Female & 18-24 & 16-30hrs & Neutral & Visual Style & No & Very intuitive
& No & Neutral & Very clearly & Visuals;The environment felt like I was
imprisoned & Somewhat detracted from the experience & 6 \\
Female & 18-24 & 1-5hrs & Somewhat disengaging & Pacing of GamePlay & No
& Somewhat intuitive & No & Frustratingly difficult & Somewhat unclearly
& Text/dialogue & Did not effect experience & 5 \\
Male & 18-24 & 16-30hrs & Somewhat engaging & Visual Style & Yes & Very
intuitive & No & Just Right & Very clearly &
Visuals;Puzzles;Text/dialogue & Enhanced the experience greatly & 8 \\
Female & 25-34 & 6-15hrs & Somewhat engaging & Pacing of GamePlay & Yes
& Somewhat intuitive & No & Neutral & Somewhat clearly & Puzzles;Audio &
Somewhat enhanced the experience & 6 \\
Female & 35-44 & 1-5hrs & Very engaging & Pacing of GamePlay & Yes &
Somewhat intuitive & No & Just Right & Somewhat clearly &
Audio;Text/dialogue & Enhanced the experience greatly & 8 \\
Male & 18-24 & 16-30hrs & Very engaging & Visual Style & Yes & Very
intuitive & No & Neutral & Very clearly & Visuals;Text/dialogue &
Enhanced the experience greatly & 9 \\
Female & 18-24 & 16-30hrs & Very engaging & Visual Style & Yes & Very
intuitive & No & Too Easy & Very clearly & Visuals;Puzzles & Enhanced
the experience greatly & 10 \\
Male & 18-24 & 31-50hrs & Somewhat engaging & Visual Style & No & Very
intuitive & No & Frustratingly difficult & Somewhat unclearly &
Visuals;Text/dialogue & Did not effect experience & 3 \\
Female & 18-24 & 31-50hrs & Very engaging & Visual Style & Yes & Very
intuitive & No & Just Right & Very clearly &
Visuals;Puzzles;Audio;Text/dialogue & Enhanced the experience greatly &
10 \\
Male & 25-34 & 16-30hrs & Somewhat engaging & Audio Effects & Yes &
Somewhat intuitive & No & Just Right & Very clearly &
Visuals;Puzzles;Environment was top notch & Enhanced the experience
greatly & 9 \\
Male & 45-54 & 1-5hrs & Somewhat engaging & Narrative Tone & Yes &
Neutral & Yes(Please Specify) & Just Right & Somewhat clearly &
Visuals;Text/dialogue;Intro cinematic & Somewhat enhanced the experience
& 7 \\
\end{longtable}



\end{document}
